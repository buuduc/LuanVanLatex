In order to separate the distortion part from dilatation, it is necessary to introduce the so-called reduced invariants, $ J_{1}, J_{2} $, and $ J_{3} $, defined by
\begin{equation}
 J_{1}=I_{1} I_{3}^{-1 / 3}, \quad J_{2}=I_{2} I_{3}^{-2 / 3}, \quad J_{3}=I_{3}^{1 / 2} 
\end{equation}
where $ I_{1}, I_{2} $, and $ I_{3} $ are the three invariants of the right Cauchy-Green deformation tensor C. It can be easily verified that $ J_{1} $ and $ J_{2} $ are constant under pure dilatation; they are only related to distortion, while $ J_{3} $ is related to dilatation. Of course, when the material is purely incompressible, the reduced invariants are the same with the invariants of $ \mathbf{C} $.

Using the reduced invariants, it is possible to separate the distortion effect from dilatation in defining the strain energy density, as
\begin{equation}
 W\left(J_{1}, J_{2}, J_{3}\right)=W_{1}\left(J_{1}, J_{2}\right)+W_{2}\left(J_{3}\right) 
\end{equation}
where $ W_{1}\left(J_{1}, J_{2}\right) $ is the distortional strain energy density and $ W_{2}\left(J_{3}\right) $ is the dilatational strain energy density. The distortional energy density can be defined using Eq. (\ref{eqn:3.105}) by substituting the reduced invariants for the original invariants. An example of the dilatational energy density is related to the bulk modulus of the material as
\begin{equation}
W_{2}\left(J_{3}\right)=\frac{K}{2}\left(J_{3}-1\right)^{2}
\end{equation}

where $ K $ is the bulk modulus. The relationship between the bulk modulus and Lame's constants for an isotropic material can be written as
\begin{equation}
K=\lambda+\frac{2}{3} \mu
\end{equation}

The above relation is valid for linear elastic materials. For general nearly incompressible materials, a large value of the bulk modulus is used-two or three orders of magnitude larger than material parameters in the distortional part. The material becomes incompressible as the bulk modulus approaches infinity.
