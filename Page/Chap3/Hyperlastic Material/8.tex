\subsection{Nearly Incompressible Hyperelasticity}
Incompressibility of a material can cause many difficulties in the constitutive
relation, especially when it is combined with nonlinearities such as large displace-
ments, large strains, and contact. Perfect incompressibility, which corresponds to
Poisson’s ratio of one-half, is an idealization to make modeling more amenable for
obtaining closed-form solutions. In the real world, natural as well as filled rubbers
are slightly compressible, thereby, facilitating development of algorithms for nearly
incompressible behavior of elastomers. “Near-incompressibility” means that
Poisson’s ratio is not exactly one-half, but close to it. For example, 0.49 or higher
values are often used for the nearly incompressible behavior of elastomers.

As discussed previously, the hydrostatic pressure portion of stress causes volume
change (dilatation). However, if the material is incompressible, the volume remains
constant for different values of pressure. In other words, stress cannot be obtained
by differentiating the strain energy density because the hydrostatic pressure portion
of stress cannot be determined from deformation.

It has been observed from experiments that many rubberlike materials show
nearly incompressible properties. It means that only a small volume change occurs
under a large hydrostatic pressure. In such materials, the near-incompressibility can
be imposed by using a large bulk modulus, which relates hydrostatic pressure to
volumetric strain. Since the material is stiff in dilatation and soft in distortion, it is
necessary to separate these two parts in order to reduce numerical difficulties
associated with a large difference in stiffness. This has to be done in the level of
strain energy density.

In the previous section, it is discussed that the third invariant $I _3$ is related to
dilatation, while the other two invariants, $I_ 1$ and $I _2$ , are related to distortion.
However, $I _1$ and $I _2$ do not remain constant during dilatation