\chapter*{\centering Abstract}% Publications page text

Rubber, plastic, their similar materials and many other polymer materials are hyper elastic materials. We can see the use of these materials increasingly common in engineering with familiar products such as tires, car front-end covers, products made of plastic and rubber, etc. So solving the problems of hyper-elastic materials is essential in engineering. Solving these problems is a difficult engineering task. Since hyper-elastic materials have a stress-and-strain relationship that is nonlinear. Therefore, a suitable and effective method is required. And one of the most commonly used methods in engineering is the finite element method (FEM). . This paper discusses solving the contact problem between two elastic materials. In addition to using FEM, we also use high-order elements to deal with advantages such as: less element usage, more accurate results, geometrical flexibility....The contribution of the paper is that Matlab programs can calculate and simulate the specific contact problem between two hyper-elastic materials. To increase reliability, the obtained results are verified with the solutions given by FEA program. In summary, this paper said that it is high feasibility to use high-order elements in computational programming. With its advantages, the high-order elements are used in many contact problems requiring high accuracy. 