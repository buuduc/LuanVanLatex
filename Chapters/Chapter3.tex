% Chapter 3

\chapter{ Finite element algorithms for contact problems} % Main chapter title

\label{Chapter3} % For referencing the chapter elsewhere, use \ref{Chapter1} 

%----------------------------------------------------------------------------------------

\section{Finite element method for solid mechanics problems \parencite{ref2}\parencite{ref3}}
%----------------------------
\subsection{Problem statement}
The displacements along coordinate axes x, y and z are defined by the displacement vector $\{u\}$ 

\begin{equation}
    \label{eqn:u}
    \{u\}=\left\{\begin{array}{lll}u & v & w \end{array}\right\} 
   \end{equation}
   Six different strain components are able to place in the strain vector $\{\varepsilon\}$:

\begin{equation}
    \{\varepsilon\}=\{\varepsilon_x,\varepsilon_y,\varepsilon_z,\varepsilon_{xy},\varepsilon_{yz},\varepsilon_{xz} \}
    % \label{eqn:epsilon}
\end{equation}
Which are related to strains for elastic body by the Hook’s law:

\begin{equation}
    \begin{array}{l}
        \{ \sigma \}  = [E]\left\{ {{\varepsilon ^e}} \right\} = [E]\left( {\{ \varepsilon \}  - \left\{ {{\varepsilon ^t}} \right\}} \right)\\
        \left\{ {{\varepsilon ^t}} \right\} = \{ \alpha T\quad \alpha T\quad \alpha T\quad 0\quad 0\quad 0\} 
    \end{array}
   \end{equation}
Here $\{\varepsilon^e \}$ is the elastic part of strains; $\{\varepsilon^t\}$ is the thermal part of strains; $\alpha$ is the coefficient of thermal expansion; T is temperature. The elasticity matrix [E] has the following appearance:

\begin{equation}
    [E] = \left[ {\begin{array}{*{20}{c}}
        {\lambda  + 2\mu }&\lambda &\lambda &0&0&0\\
        \lambda &{\lambda  + 2\mu }&\lambda &0&0&0\\
        \lambda &\lambda &{\lambda  + 2\mu }&0&0&0\\
        0&0&0&\mu &0&0\\
        0&0&0&0&\mu &0\\
        0&0&0&0&0&\mu 
        \end{array}} \right] N
\end{equation}

Where $\lambda$ and $\mu$ are elastic Lame constants which can be expressed through the Young’s modulus E and Poisson’s ratio $\nu$ :

\begin{equation} 
    \label{eq1}
    \begin{split}
        \lambda  &= \frac{{vE}}{{(1 + v)(1 - 2v)}}\\
        \mu  &= \frac{E}{{2(1 + v)}}
    \end{split}
    \end{equation}
The purpose of finite element solution of elastic problem is used to find such displacement field which provides minimum to the functional of total potential energy $\Pi$:
    \begin{equation}
 \Pi=\int_{V} \frac{1}{2}\left\{\varepsilon^{e}\right\}^{T}\{\sigma\} d V-\int_{V}\{u\}^{T}\left\{p^{v}\right\} d V-\int_{S}\{u\}^{T}\left\{p^{s}\right\} d S 
\end{equation}

